\documentstyle{article}

\setlength{\parindent}{0in}
\setlength{\parskip}{.2in}


\font\tenu=cmu10 % unslanted italic
\setlength{\textwidth}{6.5 in}
\setlength{\oddsidemargin}{0 in}
\setlength{\evensidemargin}{0 in}
\setlength{\topmargin}{-37 pt}
\setlength{\textheight}{9 in}


\pagestyle{headings}

\newcommand{\pfc}{$P_{fc}$}

\newcommand{\example}
   {\begin{quote}
    \small
    \setlength{\rightmargin}{0in}
    \begin{verbatim}}

\newcommand{\figline}{\centerline{\rule{\linewidth}{1pt}}}


\newenvironment{compactitemize}
    {\begin{itemize} \setlength{\itemsep}{-.05in} }{\end{itemize}}

% \newcommand{\gloss}[1] {\bigskip {\large\bf {#1}} \smallskip \\ \figline \bigskip}

\newcommand{\gloss}[1] {\subsubsection*{#1}}

\newcommand{\prolog}[1]{{\tt\bf {#1}}}

\begin{document}
\bibliographystyle{plain}
\title{\pfc\ User Manual}
\author{
  Tim Finin\\
  Computer Science and Electrical Engineering\\
  University of Maryland Baltimore County\\
  1000 Hilltop Circle\\
  Baltimore MD 21250\\
  finin@umbc.edu\\
  http://umbc.edu/~finin/
}

\date{August 1999}
\maketitle

\begin{abstract}

The \pfc\ system is a package that provides a forward reasoning
capability to be used together with conventional Prolog programs.  The
\pfc\ inference rules are Prolog terms which are asserted as clauses
into the regular Prolog database.  When new facts or forward reasoning
rules are added to the Prolog database (via a special predicate add/1,
forward reasoning is triggered and additional facts that can be
deduced via the application of the forward chaining rules are also
added to the database.  A simple justification-based truth-maintenance
system is provided as well as simple predicates to explore the
resulting proof trees.  
\end{abstract}


%\tableofcontents
% \clearpage
\section{Introduction}

Prolog, like most logic programming languages, offers backward
chaining as the only reasoning scheme.  It is well known that sound
and complete reasoning systems can be built using either exclusive
backward chaining or exclusive forward chaining \cite{Nilsson80}.
Thus, this is not a theoretical problem.  It is also well understood
how to ``implement'' forward reasoning using an exclusively backward
chaining system and vice versa.  Thus, this need not be a practical
problem.  In fact, many of the logic-based languages developed for AI
applications \cite{DUCK,MRS,Petrie88,Fritzson88a} allow one to build
systems with both forward and backward chaining rules.

There are, however, some interesting and important issues which need
to be addresses in order to provide the Prolog programmer with a
practical, efficient, and well integrated facility for forward
chaining.  This paper describes such a facility, \pfc\ , which we have
implemented in standard Prolog.

The \pfc\ system is a package that provides a forward reasoning
capability to be used together with conventional Prolog programs.  The
\pfc\ inference rules are Prolog terms which are asserted as facts
into the regular Prolog database.  For example, Figure
\ref{fig:pfcrules} shows a file of \pfc\ rules and facts which are
appropriate for the ubiquitous kinship domain.

\begin{figure}[bhp]
\figline
\small
\begin{verbatim}
spouse(X,Y) <=> spouse(Y,X).
spouse(X,Y),gender(X,G1),{otherGender(G1,G2)}
     =>gender(Y,G2).
gender(P,male) <=> male(P).
gender(P,female) <=> female(P).
parent(X,Y),female(X) <=> mother(X,Y).
parent(X,Y),parent(Y,Z) => grandparent(X,Z).
grandparent(X,Y),male(X) <=> grandfather(X,Y).
grandparent(X,Y),female(X) <=> grandmother(X,Y).
mother(Ma,Kid),parent(Kid,GrandKid)
      =>grandmother(Ma,GrandKid).
grandparent(X,Y),female(X) <=> grandmother(X,Y).
parent(X,Y),male(X) <=> father(X,Y).
mother(Ma,X),mother(Ma,Y),{X\==Y}
     =>sibling(X,Y).
\end{verbatim}
\caption[Pfc Rules]{{\bf Examples of \pfc\ rules which represent common kinship relations}}
\label{fig:pfcrules}
\figline
\end{figure}



The rest of this manual is structured as follows.  The next section
provides an informal introduction to the \pfc\ language.  Section
three describes the predicates through which the user calls \pfc\.
The final section gives several longer examples of the use of \pfc\

\subsection*{Getting and installing \pfc\ }

Look for \pfc\ on ftp.cs.umbc.edu in /pub/pfc/.


\section{An Informal Introduction to the  \pfc\ language}

This section describes \pfc\ .  We will start by introducing the
language informally through a series of examples drawn from the domain
of kinship relations.  This will be followed by an example and a
description of some of the details of its current implementation.

\subsection*{Overview}

The \pfc\ package allows one to define forward chaining rules and to
add ordinary Prolog assertions into the database in such a way as to
trigger any of the \pfc\ rules that are satisfied.  An example of a
simple \pfc\ rule is:
\example
gender(P,male) => male(P)
\end{verbatim}\end{quote}
This rule states that whenever the fact unifying with $gender(P,male)$
is added to the database, then the fact $male(P)$ is true.  If this
fact is not already in the database, it will be added.  In any case, a
record will be made that the validity of the fact $male(P)$ depends,
in part, on the validity of this forward chaining rule and the fact
which triggered it.  To make the example concrete, if we add
$gender(john,male)$, then the fact $male(john)$ will be added to the
database unless it was already there.

In order to make this work, it is necessary to use the predicate {\em
add/1} rather than {\em assert/1} in order to assert \pfc\ rules and
any facts which might appear in the lhs of a \pfc\ rule.  

\subsection*{Compound Rules}

A slightly more complex rule is one in which the rule's left hand side
is a conjunction or disjunction of conditions:
\example
parent(X,Y),female(X) => mother(X,Y) 
mother(X,Y);father(X,Y) => parent(X,Y)
\end{verbatim}\end{quote}
The first rule has the effect of adding the assertion $mother(X,Y)$ to
the database whenever $parent(X,Y)$ and $female(X)$ are simultaneously
true for some $X$ and $Y$.  Again, a record will be kept that
indicates that any fact $mother(X,Y)$ added by the application of this
rule is justified by the rule and the two triggering facts.  If any
one of these three clauses is removed from the database, then all
facts solely dependent on them will also be removed.  Similarly, the
second example rule derives the parent relationship whenever either
the mother relationship or the father relationship is known.

In fact, the lhs of a \pfc\ rule can be an arbitrary conjunction or
disjunction of facts.  For example, we might have a rule like:
\example
P, (Q;R), S => T
\end{verbatim}\end{quote}
\pfc\ handles such a rule by putting it into conjunctive normal form.
Thus the rule above is the equivalent to the two rules:
\example
P,Q,S => T
P,R,S => T
\end{verbatim}\end{quote}

\subsection*{Bi-conditionals}

\pfc\ has a limited ability to express bi-conditional rules, such as:
\example
mother(P1,P2) <=> parent(P1,P2), female(P1).
\end{verbatim}\end{quote}
In particular, adding a rule of the form {\tt P<=>Q} is the equivalent
to adding the two rules {\tt P=>Q} and {\tt Q=>P}.  The limitations on
the use of bi-conditional rules stem from the restrictions that the
two derived rules be valid horn clauses.  This is discussed in a later
section.

\subsection*{Backward-Chaining \pfc\ Rules}

\pfc\ includes a special kind of backward chaining rule which is used
to generate all possible solutions to a goal that is sought in the
process of forward chaining.  Suppose we wished to define the {\em
ancestor} relationship as a \pfc\ rule.  This could be done as:
\example
parent(P1,P2) => ancestor(P1,P2).
parent(P1,P2), ancestor(P2,P3) => ancestor(P1,P3).
\end{verbatim}\end{quote}
However, adding these rules will generate a large number of
assertions, most of which will never be needed.  An alternative is to
define the {\em ancestor} relationship by way of backward chaining
rules which are invoked whenever a particular ancestor relationship is
needed.  In \pfc\, this need arises whenever facts matching the
relationship are sought while trying a forward chaining rule.
\example
ancestor(P1,P2) <= {\+var(P1)}, parent(P1,X), ancestor(X,P2).
ancestor(P1,P2) <= {var(P1),\+var(P2)}, parent(X,P2), ancestor(P2,X).
\end{verbatim}\end{quote}

\subsection*{Conditioned Rules}

It is sometimes necessary to add some further condition on a rule.
Consider a definition of sibling which states:
\begin{quote}
Two people are siblings if they have the same mother and the same
father.  No one can be his own sibling.
\end{quote}
This definition could be realized by the following \pfc\ rule
\example
mother(Ma,P1), mother(Ma,P2), {P1\==P2},
  father(Pa,P1), father(Pa,P2)
   =>  sibling(P1,P2).
\end{verbatim}\end{quote}
Here we must add a condition to the lhs of the rule which states the
the variables $P1$ and $P2$ must not unify.  This is effected by
enclosing an arbitrary Prolog goal in braces.  When the goals to the
left of such a bracketed condition have been fulfilled, then it will
be executed.  If it can be satisfied, then the rule will remain
active, otherwise it will be terminated.

\subsection*{Negation}

We sometimes want to draw an inference from the absence of some
knowledge.  For example, we might wish to encode the default rule that
a person is assumed to be male unless we have evidence to the
contrary:
\example
person(P), ~female(P) => male(P).
\end{verbatim}\end{quote}
A lhs term preceded by a $\sim$ is satisfied only if {\em no} fact in
the database unifies with it.  Again, the \pfc\ system records a
justification for the conclusion which, in this case, states that it
depends on the absence of the contradictory evidence.  The behavior of
this rule is demonstrated in the following dialogue:
\example
?- add(person(P), ~female(P) => male(P)).
yes
?- add(person(alex)).
yes
?- male(alex).
yes
?- add(female(alex)).
yes
?- male(alex)
no
\end{verbatim}\end{quote}

As a slightly more complicated example, consider a rule which states
that we should assume that the parents of a person are married unless
we know otherwise.  Knowing otherwise might consist of either knowing
that one of them is married to a yet another person or knowing that
they are divorced.  We might try to encode this as follows:
\example
parent(P1,X), 
parent(P2,X),
{P1\==P2},
~divorced(P1,P2),
~spouse(P1,P3),
{P3\==P2},
~spouse(P2,P4),
{P4\==P1}
  =>
spouse(P1,P2).
\end{verbatim}\end{quote}
Unfortunately, this won't work.  The problem is that the conjoined
condition
\example
~spouse(P1,P3),{P3\==P2}
\end{verbatim}\end{quote}
does not mean what we want it to mean - that there is no $P3$ distinct
from $P2$ that is the spouse of $P1$.  Instead, it means that $P1$ is
not married to any $P3$.  We need a way to move the qualification 
\verb+{P3\==P2}+ inside the scope of the negation.  To achieve this, we
introduce the notion of a qualified goal.  A lhs term $P/C$, where P
is a positive atomic condition, is true only if there is a database
fact unifying with $P$ and condition $C$ is satisfiable.  Similarly, a
lhs term $\sim P/C$, where P is a positive atomic condition, is true
only if there is no database fact unifying with $P$ for which
condition $C$ is satisfiable.  Our rule can now be expressed as
follows:
\example
parent(P1,X), 
  parent(P2,X)/(P1\==P2),
  ~divorced(P1,P2),
  ~spouse(P1,P3)/(P3\==P2),
  ~spouse(P2,P4)/(P4\==P1)
  =>
  spouse(P1,P2).
\end{verbatim}\end{quote}

\subsection*{Procedural Interpretation}

Note that the procedural interpretation of a \pfc\ rule is that the
conditions in the lhs are checked {\em from left to right}.  One
advantage to this is that the programmer can chose an order to the
conditions in a rule to minimize the number of partial instantiations.
% include an example here
Another advantage is that it allows us to write rules like the
following:
\example
at(Obj,Loc1),at(Obj,Loc2)/{Loc1\==Loc2} 
   => {remove(at(Obj,Loc1))}.
\end{verbatim}\end{quote}
Although the declarative reading of this rule can be questioned, its
procedural interpretation is clear and useful:
\begin{quotation}
If an object is known to be at location $Loc1$ and an assertion is
added that it is at some location $Loc2$, distinct from $Loc1$, then
the assertion that it is at $Loc1$ should be removed.
\end{quotation}

\subsection*{The Right Hand Side}

The examples seen so far have shown a rules rhs as a single
proposition to be ``added'' to the database.  The rhs of a \pfc\ rule
has some richness as well.  The rhs of a rule is a conjunction of
facts to be ``added'' to the database and terms enclosed in brackets
which represent conditions/actions which are executed.  As a simple
example, consider the conclusions we might draw upon learning that one
person is the mother of another:
\example
mother(X,Y) =>
  female(X),
  parent(X,Y),
  adult(X).
\end{verbatim}\end{quote}

As another example, consider a rule which detects bigamists and sends
an appropriate warning to the proper authorities:
\example
spouse(X,Y), spouse(X,Z), {Y\==Z} => 
   bigamist(X), 
   {format("~N~w is a bigamist, married
      to both ~w and ~w~n",[X,Y,Z])}.
\end{verbatim}\end{quote}
Each element in the rhs of a rule is processed from left to right ---
assertions being added to the database with appropriate support and
conditions being satisfied.  If a condition can not be satisfied, the
rest of the rhs is not processed.

We would like to allow rules to be expressed as bi-conditional in so
far a possible.  Thus, an element in the lhs of a rule should have an
appropriate meaning on the rhs as well.  What meaning should be
assigned to the conditional fact construction (e.g. $P/Q$) which can
occur in a rules lhs?  Such a term in the rhs of a rule is
interpreted as a {\em conditioned assertion}. Thus the assertion $P/Q$
will match a condition $P\prime$ in the lhs of a rule only if $P$ and
$P\prime$ unify and the condition $Q$ is satisfiable.  For example,
consider the rules that says that an object being located at one place
is reason to believe that it is not at any other place:
\example
at(X,L1) => not(at(X,L2))/L2\==L1
\end{verbatim}\end{quote}
Note that a {\em conditioned assertion} is essentially a Horn clause.
We would express this fact in Prolog as the backward chaining rule:
\example
not(at(X,L2)) :- at(X,L1),L1\==L2.
\end{verbatim}\end{quote}
The difference is, of course, that the addition of such a conditioned
assertion will trigger forward chaining whereas the assertion of a new
backward chaining rule will not.

\subsection*{The Truth Maintenance System}

As discussed in the previous section, a forward reasoning system has
special needs for some kind of {\em truth maintenance system}.  The
\pfc\ system has a rather straightforward TMS system which records
justifications for each fact deduced by a \pfc\ rule.  Whenever a fact
is removed from the database, any justifications in which it plays a
part are also removed.  The facts that are justified by a removed
justification are checked to see if they are still supported by some
other justifications.  If they are not, then those facts are also
removed.

Such a TMS system can be relatively expensive to use and is not needed
for many applications.  Consequently, its use and nature are optional
in \pfc\ and are controlled by the predicate $pfcTmsMode/1$.  The
possible cases are three:
\begin{itemize}

\item $pfcTmsMode(full)$ - The fact is removed unless it has {\em well
found\-ed support} (WFS).  A fact has WFS if it is supported by the
$user$ or by $God$ or by a justification all of whose justificees have
WFS\footnote{Determining if a fact has WFS requires detecting local
cycles - see \cite{mcdermott85} for an introduction}.

\item $pfcTmsMode(local)$ - The fact is removed if it has no
supporting justifications.

\item $pfcTmsMode(none)$ -  The fact is never removed. 
\end{itemize}

A fact is considered to be supported by $God$ if it is found in the
database with no visible means of support.  That is, if \pfc\
discovers an assertion in the database that can take part in a forward
reasoning step, and that assertion is not supported by either the user
or a forward deduction, then a note is added that the assertion is
supported by $God$.  This adds additional flexibility in interfacing
systems employing \pfc\ to other Prolog applications.

For some applications, it is useful to be able to justify actions
performed in the rhs of a rule.  To allow this, \pfc\ supports the
idea of declaring certain actions to be {\em undoable} and provides
the user with a way of specifying methods to undo those actions.
Whenever an action is executed in the rhs of a rule and that action is
undoable, then a record is made of the justification for that action.
If that justification is later invalidated (e.g. through the
retraction of one of its justificees) then the support is checked for
the action in the same way as it would be for an assertion.  If the
action does not have support, then \pfc\ trys each of the methods it
knows to undo the action until one of them succeeds.

In fact, in \pfc\ , one declares an action as undoable just by
defining a method to accomplish the undoing.  This is done via the
predicate $pfcUndo/2$.  The predicate $pfcUndo(A1,A2)$ is
true if executing $A2$ is a possible way to undo the execution of
$A1$.  For example, we might want to couple an assertional
representation of a set of graph nodes with a graphical display of
them through the use of \pfc\ rules:
\example
at(N,XY) => {displayNode(N,XY)}.
arc(N1,N2) => {displayArc(N1,N2}.

pfcUndo(displayNode(N,XY),eraseNode(N,XY)).
pfcUndo(displayArc(N1,N2),eraseArc(N1,N2)).
\end{verbatim}\end{quote}

\subsection*{Limitations}

The \pfc\ system has several limitations, most of which it inherits
from its Prolog roots.  One of the more obvious of these is that \pfc\
rules must be expressible as a set of horn clauses.  The practical
effect is that the rhs of a rule must be a conjunction of terms which
are either assertions to be added to the database or actions to be
executed.  Negated assertions and disjunctions are not permitted,
making rules like
\example
parent(X,Y) <=> mother(X,Y);father(X,Y)
male(X) <=> ~female(X)
\end{verbatim}\end{quote}
ill-formed. 

Another restrictions is that all variables in a \pfc\ rule have
implicit universal quantification.  As a result, any variables in the
rhs of a rule which remain uninstantiated when the lhs has been fully
satisfied retain their universal quantification.  This prevents us
from using a rule like
\example
father(X,Y), parent(Y,Z) 
    <=> grandfather(X,Z).
\end{verbatim}\end{quote}
with the desired results.  If we do add this rule and assert {\em
grandfather(john,mary)}, then \pfc\ will add the two independent
assertions {\em father(john,\_)} (i.e. ``John is the father of
everyone'') and {\em parent(\_,mary)} (i.e. ``Everyone is Mary's
parent'').

Another problem associated with the use of the Prolog database is that
assertions containing variables actually contain ``copies'' of the
variables.  Thus, when the conjunction
\example
add(father(adam,X)), X=able
\end{verbatim}\end{quote}
is evaluated, the assertion \verb#father(adam,_G032)# is added to the
da\-ta\-base, where \_G032 is a new variable which is distinct from X.  As
a consequence, it is never unified with {\em able}.


\section{Predicates}
\subsection{Manipulating the Database}

\gloss{add(+P)}

The fact or rule P is added to the database with support coming from
the user.  If the fact already exists, an additional entry will not be
made (unlike Prolog).  If the facts already exists with support from
the user, then a warning will be printed if $pfcWarnings$ is true.
Add/1 always succeeds.

\gloss{pfc(?P)}

The predicate $pfc/1$ is the proper way to access terms in the \pfc\
database.  \prolog{pfc(P)} succeeds if \prolog{P} is a term in the
current pfc database after invoking any backward chaining rules or is
provable by Prolog.

\gloss{rem(+P)}

The first fact (or rule) unifying with $P$ has its user support
removed.  $rem/1$ will fail if no there are no \pfc\ added facts or
rules in the database which match.  If removing the user support from
a fact leaves it unsupported, then it will be removed from the
database.

\gloss{rem2(+P}

The first fact (or rule) unifying with $P$ will be removed from the
database even if it has valid justifications.  $rem/1$ will fail if no
there are no \pfc\ added facts or rules in the database which match.
If removing the user support from the fact leaves it unsupported, then
it will be removed from the database.  If the fact still has valid
justifications, then a \pfc\ warning message will be printed and the
justifications removed.

\gloss{pfcReset}.

Resets the \pfc\ database by trying to retract all of the prolog
clauses which were added by calls to add or by the forward chaining
mechanism.

% \gloss{pfc\_save\\	pfc\_save(+Id)\\	pfc\_file\_save(+FileSpec)}
% \gloss{pfc\_restore\\	pfc\_restore(+Id)\\	pfc\_file\_restore(+FileSpec)}


\gloss{Term expansions}

\pfc\ defines term expansion procedures for the operators {\em =>},
{\em <=} and {\em <=>} so that you can have things like the following
in a file to be consulted
\example
foo(X) => bar(X).
=> foo(1).
\end{verbatim}\end{quote}

The result will be an expansion to:
\example
:- add((foo(X) => bar(X)).
:- add(foo(1)).
\end{verbatim}\end{quote}


% \gloss{pfcLoad(+File)}
% \gloss{pfcCompile(+File)}

\subsection{Control Predicates}

This section describes predicates to control the forward chaining
search strategy and truth maintenance operations.

\gloss{pfcSearch(P)}

This predicate is used to set the search strategy that \pfc\ uses in
doing forward chaing.  The argument should be one of
{direct,depth,breadth}.

\gloss{pfcTmsMode(Mode)}

This predicate controls the method used for truth maintenaance.  The
three options are {none,local,cycles}. Calling pfcTmsMode with an
instantiated argument will set the mode to that argument. 
\begin{itemize}

\item {\bf none} means that no truth maintenance will be done.

\item {\bf local} means that limited truth maintenance will be done. Specifically, no cycles will be checked.

\item {\bf cycles} means that full truth maintenance will be done,
including a check that all facts are well grounded.

\end{itemize}


\gloss{pfcHalt}

Immediately stop the forward chainging process.

\gloss{pfcRun}

Continue the forward chainging process.

\gloss{pfcStep}

Do one iteration of the forward chainging process.

\gloss{pfcSelect(P)}

Select next fact for forward chaining (user defined)

\gloss{pfcWarnings\\pfcNoWarnings}




\subsection{The TMS}

The following predicates are used to access the tms information
associated with \pfc\ facts.

\gloss{justification(+P,-J)\\
	justification(+P,-Js)}

\prolog{justification(P,J)} is true if one of the justifications for
fact P is J, where J is a list of \pfc\ facts and rules which taken
together deduce P.  Backtracking into this predicate can produce
additional justifications.  If the fact was added by the user, then
one of the justifications will be the list \prolog{[user]}.
\prolog{justifications(P,Js)} is provided for convenience.  It binds
\prolog{Js} to a list of all justifications returned by
\prolog(justification/2).  

\gloss{base(+P,-Ps)}

\gloss{assumptions(+P,-Ps)}

\gloss{pfcChild(+P,-Q)\\
	pfcChildren(+P,-Qs)}

\gloss{pfcDescendant(+P,-Q)\\
	pfcDescendants(+P,-Qs)}


\subsection{Debugging}

\gloss{pfcTrace\\
	pfcTrace(+Term)\\
	pfcTrace(+Term,+Mode)\\
	pfcTrace(+Term,+Mode,+Condition)}

This predicate causes the addition and/or removal of \pfc\ terms to be
traced if a specified condition is met. The arguments are as follows:
\begin{itemize}

\item term - Specifies which terms will be traced.  Defaults to
\prolog{\_} (i.e. all terms).

\item mode - Specifies whether the tracing will be done on the addition (i.e.
\prolog{add}, removal (i.e. \prolog{rem}) or both (i.e. \prolog{\_}) of
the term.  Defaults to \prolog{\_}.

\item condition - Specifies an additional condition which must be met
in order for the term to be traced.  For example, in order to trace
both the addition and removal of assertions of the age of people just
when the age is greater than 100, you can do
\prolog{pfcTrace(age(\_,N),\_,N>100)}.
\end{itemize}

Thus, calling \prolog{pfcTrace} will cause all terms to be traced when
they are added and removed from the database.  When a fact is added or
removed from the database, the lines
\example
1
2
\end{verbatim}\end{quote}
are displayed, respectively.  


\gloss{pfcUntrace\\
	pfcUntrace(+Term)\\
	pfcUntrace(+Term,+Mode)\\
	pfcUntrace(+Term,+Mode,+Condition)}

The \prolog{pfcUntrace} predicate is used to stop tracing \pfc\ facts.
Calling \prolog{pfcUntrace(P,M,C)} will stop all tracing
specifications which match.  The arguments default as described above.

\gloss{pfcSpy(+Term)\\
	pfcSpy(+Term,+Mode)\\
	pfcSpy(+Term,+Mode,+Condition)}

These predicates set spypoints, of a sort.

\gloss{pfcQueue}

Displays the current queue of facts in the \pfc\ queue.

\gloss{showState} Displays the state of Pfc, including the queue, all
triggers, etc.

\gloss{pfcFact(+P)\\pfcFacts(+L)}

pfcFact(P) unifies P with a fact that has been added to the database
via \pfc\.  You can backtrac into it to find more facts.  pfcFacts(L)
unified L with a list of all of the facts asserted by add.

\gloss{pfcPrintDb\\pfcPrintFacts\\pfcPrintRules}

These predicates diaply the the entire \pfc\ database (facts and
rules) or just the facts or just the rules.


%\subsection{Miscillaneous Predicates}


   %predicates for initialization. etc.
   pfcReset/0,	% 


\section{Examples}

\subsection{Factorial and Fibonacci}

These examples show that the \pfc\ backward chaining facility can do
such standard examples as the factorial and Fibonacci functions.  

Here is a simple example of a \pfc\ backward chaining rule to compute
the Fibonacci series.
\example
fib(0,1).
fib(1,1).
fib(N,M) <= 
  N1 is N-1,
  N2 is N-2,
  fib(N1,M1),
  fib(N2,M2),
  M is M1+M2.
\end{verbatim}\end{quote}


Here is a simple example of a \pfc\ backward chaining rule to compute
the factorial function.

\example
=> fact(0,1).
fact(N,M) <=
  N1 is N-1,
  fact(N1,M1),
  M is N*M1.
\end{verbatim}\end{quote}


\subsection{Default Reasoning}

This example shows how to define a default rule.  Suppose we would
like to have a default rule that holds in the absence of contradictory
evidence. We might like to state, for example, that an we should
assume that a bird can fly unless we know otherwise.  This could be
done as:

\example
bird(X), ~not(fly(X)) => fly(X).
\end{verbatim}\end{quote}
We can, for our convenience, define a {\em default} operator which
takes a \pfc\ rule and qualifies it to make it a default rule.  This
can be done as follows:
\example
default((P => Q)),{pfcAtom(Q)} => (P, ~not(Q) => Q).
\end{verbatim}\end{quote}
where \prolog{pfcAtom(X)} holds if \prolog{X} is a ``logical atom''
with respect to \pfc\ (i.e . not a conjunction, disjunction, negation,
etc).


One we have defined this, we can use it to state that birds fly by
default, but penguins do not.
\example
% birds fly by default.
=> default((bird(X) => fly(X))).

isa(C1,C2) =>
  % here's one way to do an isa hierarchy.
  {P1 =.. [C1,X],
    P2 =.. [C2,X]},
  (P1 => P2).

=> isa(canary,bird).
=> isa(penguin,bird).

% penguins do not fly.
penguin(X) => not(fly(X)).

% chilly is a penguin.
=> penguin(chilly).

% tweety is a canary.
=> canary(tweety).
\end{verbatim}\end{quote}


\subsection{KR example}

isa hierarchy. roles.  types. classification. etc.

\subsection{Maintaining Functional Dependencies}

\label{sec:funcdep}

One useful thing that \pfc\ can be used for is to automatically
maintain function Dependencies in the light of a dynamic database of
fact. The builtin truth maintenance system does much of this.
However, it is often useful to do more.  For example, suppose we want
to maintain the constraint that a particular object can only be
located in one place at a given time.  We might record an objects
location with an assertion \prolog{at(Obj,Loc)} which states that the
current location of the object \prolog{Obj} is the location
\prolog{Loc}.  

Suppose we want to define a \pfc\ rule which will be triggered
whenever an \prolog{at/2} assertion is made and will remove any
previous assertion about the same object's location.  Thus to reflect
that an object has moved from location A to location B, we need merely
add the new information that it is at location B.  If we try to do
this with the \pfc\ rule:
\example
at(Obj,Loc1),
at(Obj,Loc2),
{Loc1\==Loc2}
=> 
~at(Obj,Loc1).
\end{verbatim}\end{quote}
we may or may not get the desired result.  This rule will in fact
maintain the constraint that the database have at most one
\prolog{at/2} assertion for a given object, but whether the one kept
is the old or the new depends on the particular search strategy being
used by \pfc\.  In fact, under the current default strategy, the new
assertion will be the one retracted.

We can achieve the desired result with the following rule:
\example
at(Obj,NewLoc), 
{at(Obj,OldLoc), OldLoc\==NewLoc}
  =>
  ~at(Obj,OldLoc).
\end{verbatim}\end{quote}
This rule causes the following behavior. Whenever a new assertion
\prolog{at(O,L)} is made, a Prolog search is made for an assertion that
object O is located at some other location.  If one is found, then it
is removed.

We can generalize on this rule to define a meta-predicate
\prolog{function(P)} which states that the predicate whose name is
\prolog{P} represents a function.  That is, \prolog{P} names a
relation of arity two whose first argument is the domain of the
function and whose second argument is the function's range.  Whenever
an assertion \prolog{P(X,Y)} is made, any old assertions matching
\prolog{P(X,\_)} are removed.  Here is the \pfc\ rule:
\example
function(P) =>
  {P1 =.. [P,X,Y],
   P2 =.. [P,X,Z]},
  (P1,{P2,Y\==Z} => ~P2).
\end{verbatim}\end{quote}

We can try this with the following results:
\example
| ?- add(function(age)).
Adding (u) function(age)
Adding age(A,B),{age(A,C),B\==C}=> ~age(A,C)
yes

| ?- add(age(john,30)).
Adding (u) age(john,30)
yes

| ?- add(age(john,31)).
Adding (u) age(john,31)
Removing age(john,30).
yes
\end{verbatim}\end{quote}

Of course, this will only work for functions of exactly one argument,
which in Prolog are represented as relations of arity two.  We can
further generalize to functions of any number of arguments (including
zero), with the following rule:

\example

function(Name,Arity) =>
  {functor(P1,Name,Arity),
   functor(P2,Name,Arity),
   arg(Arity,P1,PV1),
   arg(Arity,P2,PV2),
   N is Arity-1,
   merge(P1,P2,N)},
  (P1,{P2,PV1\==PV2} => ~P2).


merge(_,_,N) :- N<1.
merge(T1,T2,N) :-
  N>0,
  arg(N,T1,X),
  arg(N,T2,X),
  N1 is N-1,
  merge(T1,T2,N1).

\end{verbatim}\end{quote}
The result is that adding the fact \prolog{function(P,N)} declares P
to be the name of a relation of arity N such that only the most recent
assertion of the form $P(a_{1},a_{2},\ldots,a_{n-1},a_{n})$ for a
given set of constants $a_{1},\ldots,a_{n-1}$ will be in the database.
The following examples show how we might use this to define a
predicate \prolog{current\_president/1} that identifies the current
U.S. president and \prolog{governor/3} that relates state, a year and
the name of its governor.
\example
% current_president(Name)
| ?- add(function(current_president,1)).
Adding (u) function(current_president,1)
Adding current_president(A),
       {current_president(B),A\==B}
	=> 
        ~current_president(B)
yes

| ?- add(current_president(reagan)).
Adding (u) current_president(reagan)
yes

| ?- add(current_president(bush)).
Adding (u) current_president(bush)
Removing current_president(reagan).
yes

% governor(State,Year,Governor)
| ?- add(function(governor,3)).
Adding (u) function(governor,3)
Adding governor(A,B,C),{governor(A,B,D),C\==D}=> ~governor(A,B,D)
yes

| ?- add(governor(pennsylvania,1986,thornburg)).
Adding (u) governor(pennsylvania,1986,thornburg)
yes

| ?- add(governor(pennsylvania,1987,casey)).
Adding (u) governor(pennsylvania,1987,casey)
yes

% oops, we misspelled thornburgh!
| ?- add(governor(pennsylvania,1986,thornburgh)).
Adding (u) governor(pennsylvania,1986,thornburgh)
Removing governor(pennsylvania,1986,thornburg).
yes
\end{verbatim}\end{quote}

\subsection{Spreadsheets}

One common kind of constraints is often found in spreadsheets in
which one value is determined from a set of other values in which the
size of the set can vary.  This is typically found in spread sheets
where one cell can be defined as the sum of a column of cells.  This
example shows how this kind of constraint can be defined in \pfc\ as well.
Suppose we have a relation \prolog{income/4} which records a person's
income for a year by source.  For example, we might have assertions like:
\example
income(smith,salary,1989,50000).
income(smith,interest,1989,500).
income(smith,dividends,1989,1200).
income(smith,consulting,1989,2000).
\end{verbatim}\end{quote}
We might also with to have a relation \prolog{total\_income/3} which
records a person's total income for each year.  Given the database
above, this should be:
\example
total_income(smith,1989,53700).
\end{verbatim}\end{quote}

One way to do this in \pfc\ is as follows:
\example
income(Person,Source,Year,Dollars) => {increment_income(Person,Year,Dollars)}.

=> pfcUndoMethod(increment_income(P,Y,D),decrement_income(P,Y,D)).

increment_income(P,Y,D) :-
  (retract(total_income(P,Y,Old)) -> New is Old+D ; New = D),
  assert(total_income(P,Y,New)).

decrement_income(P,Y,D) :-
  retract(total_income(P,Y,Old)),
  New is Old-D,
  assert(total_income(P,Y,New)).
\end{verbatim}\end{quote}
We would probably want to use the \pfc\ rule for maintaining
functional Dependencies described in Section~\ref{sec:funcdep} as
well, adding the rule:
\example
=> function(income,4).
\end{verbatim}\end{quote}

\subsection{Extended Reasoning Capability}

The truth maintenance system in \pfc\ makes it possible to do some
reasoning that Prolog does not allow.  From the facts
\begin{quote}
$p \vee q$ \\
$p \rightarrow r$ \\
$q \rightarrow r$ \\
\end{quote}
it follows that $r$ is true.  However, it is not possible to directly
encode this in Prolog so that it can be proven.  We can encode these
facts in \pfc\ and use a simple proof by contradiction strategy
embodied in the following Prolog predicate:
\example
prove_by_contradiction(P) :- P.
prove_by_contradiction(P) :-
  \+ (not(P) ; P)
  add(not(P)),
    P       -> rem(not(P))
  otherwise -> (rem(not(P)),fail).
\end{verbatim}\end{quote}
This procedure works as follows.  In trying to prove P, succeed
immediately if P is a know fact.  Otherwise, providing that
\prolog{not(P)} is not a know fact, add it as a fact and see if this
gives rise to a proof for \prolog(P).  if it did, then we have derived
a contradiction from assuming that \prolog{not(P)} is true and
\prolog{P} must be true.  In any case, remove the temporary assertion
\prolog{not(P)}.

In order to do the example above, we need to add the following rule or
\prolog{or} and a rule for general implication (encoded using the
infix operator \prolog{==>}) which generates a regular forward
chaining rule and its counterfactual rule.
\example
:- op(1050,xfx,('==>')).

(P ==> Q) => 
  (P => Q),
  (not(Q) => not(P)).

or(P,Q) => 
  (not(P) => Q),
  (not(Q) => P).
\end{verbatim}\end{quote}
With this, we can encode the problem as:
\example
=> or(p,q).
=> (p ==> x).
=> (q ==> x).
\end{verbatim}\end{quote}

When these facts are added, the following trace ensues:
\example
Adding (u) (A==>B)=>(A=>B), (not(B)=>not(A)) 
Adding (u) or(A,B)=>(not(A)=>B), (not(B)=>A) 
Adding (u) or(p,q) 
Adding not(p)=>q
Adding not(q)=>p 
Adding (u) p==>x 
Adding p=>x 
Adding not(x)=>not(p)
Adding (u) q==>x 
Adding q=>x 
Adding not(x)=>not(q)
\end{verbatim}\end{quote}

Then, we can call \prolog{prove\_by\_contradiction/1} to show that
\prolog{p} must be true:
\example
| ?- prove_by_contradiction(x).
Adding (u) not(x)
Adding not(p)
Adding q
Adding x
Adding not(q)
Adding p
Removing not(x).
Removing not(p).
Removing q.
Removing not(q).
Removing p.
Removing x.
yes
\end{verbatim}\end{quote}

% \subsection{Parsing}

%\newpage
\bibliography{/mn2/AI/ai}
\end{document}
